% !Mode:: "TeX:UTF-8"

\chapter{\LaTeX 基本介绍}
\section{\LaTeX 的由来}
\LaTeX 文字形式写作LaTeX,是一种基于TEX的排版系统,由美国电脑学家莱斯利·兰伯特在20世纪80年代初期开发,利用这种格式,即使用户没有排版和程序设计的知识也可以充分发挥由TEX所提供的强大功能,能在几天,甚至几小时内生成很多具有书籍质量的印刷品。对于生成复杂表格和数学公式,这一点表现得尤为突出。因此它非常适用于生成高印刷质量的科技和数学类文档。这个系统同样适用于生成从简单的信件到完整书籍的所有其他种类的文档。
\section{汉化}
\subsection{CCT}
最早支持简体中文的TEX是CCT,这个是中国科学院数学与系统科学研究院的张林波研究员编写。最初,由于计算机内存以及运算速度等方面的限制,需要将符合CCT格式的.ctx文件预处理之后再使用LaTeX编译,生成的.dvi文件需要后处理。 \par
在最新版的CCT中,用cct.sty代替了原来的预处理程序,与CJK结合,直接使用.tex文件,而不必再使用.ctx文件,可以用LaTeX直接编译,不再需要后处理.dvi文件。经过多年的发展,这套系统比较符合中国人的习惯,中文排版也比较符合时下中国印刷界的现行标准。
\subsection{CJK}
让\LaTeX 支持中文的另一种方法是使用CJK宏包,由德国人Werner Lemberg编写。这个宏包不仅仅支持繁简体中文、日文、朝鲜文等东亚语言,而且它也是一个多种语言支持包,另外还支持几十种其他不同的语言。
\subsection{中文套装}
现在简体中文用户使用的最广泛的TEX发行版是在Microsoft Windows平台下的CTeX中文套装,它也是最早的支持中文TEX的软件套装。hooklee制作的ChinaTeX发行版也非常不错,它集成了与TEX有关的许多软件,大大减小了初学者的安装配置困难。最有特色的是将TEX有关的命令都集成在WinTeX编辑器的按钮中,鼠标一点,即可编译。

% !Mode:: "TeX:UTF-8"

\chapter*{\centering 作者简历及在学研究成果}

\noindent 一、作者入学前简历 \par
%博士入学前简历,根据自己情况填写,备注中填写职务等。
\vspace{2ex}
\begin{tabular}{|c|c|c|}\hline
起止年月 & 学习或工作单位 & 备注 \\ \hline
XXXX年XX月至XXXX年XX月 & 在XXXX 学校XXXX 专业攻读学士学位 &  \\ \hline
XXXX年XX月至XXXX年XX月 & 在XXXX 学校XXXX 专业攻读硕士学位 &  \\ \hline
XXXX年XX月至XXXX年XX月 & 在XXXX 单位从事XXXX 岗位的工作 &  \\ \hline
\end{tabular} 

\noindent 二、在学期间从事的科研工作 \par
%(应注明课题名称、参加身份、通过时间、通过方式、评定单位等)。
\begin{enumerate}
\item 硕士博士毕业论文\LaTeX 模板的编写:\\
作为主要编辑人,从2016年3月份开始创建本项目以来一直进行修改,希望能制作一个满意的模板并且被官方认可。
\end{enumerate}
\noindent 三、在学期间所获的科研奖励 \par
%应注明奖励名称、授奖单位、授奖时间等,请填写科研方面奖励,请勿填写其他奖励信息,如不得填写三好研究生等奖励信息。)
\par
\noindent 四、在学期间发表的论文 \par
%应按照参考文献的格式来填写,包括编号。并在后面依次标明以下事项,各项之间用“.”分隔:1)标明“已发表”或“已录用”;2)是否“SCI/EI/STP/CSSCI 刊源”;3)是否被“SCI/EI/STP/CSSCI 检索”;4)检索号。第 2、3 项请标明具体检索名称)。






%盲审论文,请隐去所有可能影响盲审结果的信息,诸如作者姓名、导师姓名、作者学号等。另外在此处,研究成果中论文作者的发表文章列表中应隐去所有作者的名字,只标明论文作者是第几作者,具体如“[第二作者].论文名称.……”